\chapter{Trabajo Realizado}
\label{chap:trabajo}

\section{Características de Simulación}
El trabajo realizado a lo largo del semestre se puede dividir en dos secciones principales: la realización de simulaciones y la escritura de códigos de análisis y comparación para dichas simulaciones. Para el primer caso implica la lectura de cómo correr las simulaciones en clusters de múltiples procesadores, definir los parámetros de instalación y de simulación en sí y observar los comportamientos, tiempos de duración y modos de exportación de datos de cada simulación. Para la segunda sección se incluyen los códigos realizados y utilizados que permiten desde la lectura de los datos brutos de simulación hasta la extracción de propiedades y generación de gráficas de análisis. A continuación se explicarán las diferentes secciones y detalles del trabajo realizado, así como los resultados de dicho trabajo.

\subsection{Parámetros de instalación y compilación}
Para correr una simulación en Gadget-2 se requiere principalmente de 3 librerías previamente instaladas y de un set de condiciones iniciales sobre las cuales trabajar. Las tres librerías que se deben instalar son FFTW, GSL y MPI, todas ellas de libre distribución e instalación. FFTW es una librería, o conjunto de librerías, con códigos para la realización de transformadas de Fourier en espacio discreto, como la transformada rápida de Fourier (FFT), cuyo uso se explicó en la Sección \ref{sub:gadget}. GSL (\textit{GNU Scientific Library}) como su nombre lo dice es una librería con un gran conjunto de funciones de uso científico. Finalmente, MPI (\textit{Message Passing Interface}) es la librería de instrucciones, rutinas y métodos para la paralelización de procesos en múltiples procesadores. 

Como se explicará más a fondo en la Sección \ref{sub:simulaciones}, las simulaciones fueron realizadas no sólo en computadores personales sino que también se hizo uso del clúster de física de la Universidad de los Andes y además del clúster KIAS en Corea 
\subsection{Parámetros de la simulación}

\subsection{Simulaciones realizadas}
\label{sub:simulaciones}


\section{Códigos de Análisis}

\subsection{Estructura del \textit{Snapshot}}

\subsection{Extracción de Halos}



\subsection{Campo de densidades (CIC)}
\label{sub:CIC}

\section{Análisis y Resultados}
