\documentclass{beamer}
\usepackage[utf8]{inputenc}
\usepackage[spanish]{babel}
\usepackage{bbding}
\usetheme{Madrid}
\usecolortheme[RGB={79,79,79}]{structure}
\definecolor{darkblue}{RGB}{10,5,133}
\definecolor{ggray}{RGB}{79,79,79}
%10,5,133
\usepackage{media9}

%Justificación de todos los blocks
\usepackage{ragged2e} 
\addtobeamertemplate{block begin}{}{\justifying}

%Cambiar tamaño de los captions
\usepackage{caption}
\captionsetup{font=scriptsize,labelfont=scriptsize}

\title{Monografía}
\subtitle{Impacto de los parámetros cosmológicos en la estructura a gran escala del universo}
\author[Camilo Andrés Rivera]{Camilo Andrés Rivera 200912840\\[7mm] Asesor: Jaime Ernesto Forero.\\[3mm]}


\begin{document}

\AtBeginSection[ ]
{
	\begin{frame}
	
		\frametitle{Tabla de contenidos}
		\tableofcontents[currentsection]
		
	\end{frame}
}

\AtBeginSubsection[ ]
{
	\begin{frame}
	
		\frametitle{Tabla de contenidos}
		\tableofcontents[currentsubsection]
		
	\end{frame}
}
\begin{frame}

	\titlepage
	
\end{frame}

\begin{frame}{Agenda}
\tableofcontents
\end{frame}

%inicia presentacion
\section{Introducción y Justificación}
%================================================%
\begin{frame}
	\begin{columns}
		\begin{column}{0.4\textwidth}
			\begin{figure}[!h]
			\begin{center}
				\includegraphics[width=0.9\textwidth]{im/lss.jpg}
				\caption{LSS \footnotemark[2]} 
				\label{fig:arq1}
			\end{center}
		\end{figure}
		\end{column}
		
		\begin{column}{0.4\textwidth}
			\begin{figure}[!h]
			\begin{center}
				\includegraphics[width=1.1\textwidth]{im/DM.jpg}
				\caption{DM \footnotemark[3]} 
				\label{fig:arq2}
			\end{center}
		\end{figure}
		\end{column}
		\footnotetext[2]{Imagen Tomada de \cite{lss}}
		\footnotetext[3]{Imagen Tomada de \cite{dm}}
	\end{columns}
	
	\begin{block}{}
		\begin{itemize}
			\item La materia da cuenta de la estructura a gran escala del universo
			\item $\Omega_\Lambda,\Omega_{DM}, \Omega_0$
			\item A pesar de que domina la materia oscura, la energía oscura tiene un efecto sutil pero importante
		\end{itemize}
	\end{block}
\end{frame}
%================================================%
\begin{frame}{Motivación}
	 \begin{itemize}
	 	\item Anisotropías en CMB medidas por Plank
	 	\item Sensibilidad y margen de error en equipos de medición
	 	\item Detección de variaciones de al menos $5\%$
	\end{itemize}	 	
	
	\begin{block}{}
		\centering
		\LARGE{¿Cómo podemos medir los efectos de la energía oscura ($\Omega_\Lambda$)?}	
	\end{block}

\end{frame}
%================================================%
\section{Objetivos}
%================================================%
\begin{frame}{Objetivos}
	\begin{block}{General}
		Cuantificar el cambio de la estructura a gran escala del universo ante escenarios con diferentes parámetros cosmológicos.
	\end{block}
	\begin{block}{Específicos}
		\begin{itemize}
			\item Obtener una serie de universos simulados ante diferentes valores de parámetros cosmológicos.
			\item Extraer información acerca de los diferentes universos como la abundancia de halos de materia oscura y las distribuciones de velocidad entre pares de halos.
			\item Realizar un análisis comparativo entre los diferentes universos simulados
		\end{itemize}	
	\end{block}
\end{frame}
%================================================%
\section{Contexto del Proyecto}
%================================================%
\subsection{Contexto Teórico}
%================================================%
\begin{frame}{Evolución Universo}
	\begin{block}{}
		\begin{itemize}
			\item Inflación
			\item Fluctuaciones cuánticas en densidad
			\item Observaciones
			\begin{itemize}
				\item Expansión $H_0$
				\item Homogeneidad e isotropía
				\item Universo Plano
				\item Edad del Universo, \textit{redshift}, factor de dilatación
			\end{itemize}
			\item Parámetros cosmológicos
			\item Soluciones a Ecuaciones de Einsten: potencial gravitacional 
		\end{itemize}
	\end{block}
\end{frame}
%================================================%
\begin{frame}{Simulación en paralelo}
	\begin{block}{}
		\begin{itemize}
			\item Gadget-2
			\item N-GenIC
			\item Paralelización: MPI
			\item Octree
			\item TreePM
			\item Peano-Hilbert
		\end{itemize}
	\end{block}
	\begin{columns}
		\begin{column}{0.5\textwidth}
			\begin{figure}[!h]
			\begin{center}
				\includegraphics[width=\textwidth]{im/octree_encode}
				%\caption{$\sigma_8=0.9$} 
				\label{fig:oct}
			\end{center}
		\end{figure}
		\end{column}
		
		\begin{column}{0.5\textwidth}
			\begin{figure}[!h]
			\begin{center}
				\includegraphics[width=0.6\textwidth]{im/hilbert3d01}
				%\caption{$\sigma_8=0.7$} 
				\label{fig:pen}
			\end{center}
		\end{figure}
		\end{column}
	\end{columns}
\end{frame}
%================================================%
\subsection{Contexto Computacional}
\begin{frame}{Características de la Simulación}
	\begin{itemize}
		\item Tamaño de las simulaciones
		\begin{itemize}
			\item Cubo $\sim 500Mpc$
			\item Tiempo de evolución $\sim 13 Gyr$
			\item Número partículas $512^3$
		\end{itemize}
		\item Condiciones iniciales
		\begin{itemize}
			\item N-Genic
			\item Posiciones y velocidades
			\item $\rho$
		\end{itemize}
		\item Leyes de la física
		\item $\Omega_\Lambda$, $\Omega_0$, $H_0$
	\end{itemize}
		
\end{frame}
%================================================%
\begin{frame}{Características del Análisis}
	\begin{block}{}
	\begin{itemize}
		\item Sobredensidad
		\begin{itemize}
			\item CIC
		\end{itemize}
		\item Halos
		\begin{itemize}
			\item FoF
			\item Características de CM
			\item Identificación cruzada
			\item Diferencias y gráficas (\textit{ipython})
		\end{itemize}
	\end{itemize}
	\end{block}	
\end{frame}
%================================================%
\section{Metodología y Cronograma}
%================================================%
%\begin{frame}
%	\begin{block}{Contexto}
%	La evolución del universo y la distribución en el espacio de fase de los halos de materia ordinaria y materia oscura están fuertemente determinados por los parámetros cosmológicos $\Omega_\Lambda$, $\Omega_0$, $H_0$, entre otros.
%	La simulación de diferentes escenarios a gran escala ($\sim 1Gpc$) permite observar diferencias significativas en la estructura del universo en un análisis transversal y simultáneo.
%	\end{block}
%	\end{frame}
%\begin{frame}
%	\begin{block}{Metodología}
%	\begin{itemize}
%		\item Código libre de baja dificultad de implementación (Gadget-2 \cite{gadget})
%		\item Múltiples simulaciones ($\sim 10$) variando parámetros cosmológicos
%		\item Uso del cluster de física para realizar simulaciones en paralelo en varios procesadores ($\sim 32$)
%		\item Generación de condiciones iniciales necesarias para la evolución del universo simulado		
%		\item Hacer uso de \textit{snapshots} generados por las simulaciones para realizar análisis de las estadísticas y distribuciones 
%	\end{itemize}
%	\end{block}
%\end{frame}
%================================================%
\begin{frame}{Metodología}
	\begin{figure}
		\centering
		\includegraphics[width=0.7\textwidth]{im/Metodo}
		\caption{Metodología de Desarrollo}
		\label{fig:meto}
	\end{figure}

\end{frame}
%================================================%
%\subsection{Cronograma}
%================================================%
\begin{frame}{Cronograma}
	\centering
	\begin{figure}%
		\includegraphics[width=0.9\textwidth]{im/cronograma.png}%
		\caption{Cronograma Propuesto}%
		\label{fig:crono}%
	\end{figure}
			
\end{frame}
%================================================%
\section{Trabajo Realizado}
%================================================%
\begin{frame}{Simulaciones preliminares}
	\begin{block}{Características}
		Caja cúbica de $150 Mpc$, $128^3$ partículas, tiempo inicial de \textit{redshift} $z=63$ ($\sim 32.4 Myr$) hasta la actualidad ($\sim 13Gyr$).\\~\\
		\begin{itemize}
			\item 32 Procesadores (fiscluster)
			\item 50 snapshots
			\item $\sim 3Gb$ de almacenamiento
			\item 4 horas de simulación
			\item Variación de Sigma8 (0.9 - 0.7)
		\end{itemize}
	\end{block}	
\end{frame}
%================================================%
\begin{frame}{Simulaciones Preliminares}
	\begin{columns}
		\begin{column}{0.5\textwidth}
			\begin{figure}[!h]
			\begin{center}
				\includegraphics[width=\textwidth]{im/res1.png}
				\caption{$\sigma_8=0.9$} 
				\label{fig:res1}
			\end{center}
		\end{figure}
		\end{column}
		
		\begin{column}{0.5\textwidth}
			\begin{figure}[!h]
			\begin{center}
				\includegraphics[width=\textwidth]{im/res2.png}
				\caption{$\sigma_8=0.7$} 
				\label{fig:res2}
			\end{center}
		\end{figure}
		\end{column}
	\end{columns}
\end{frame}
%================================================%
\begin{frame}{Simulaciones Definitivas}
	\begin{block}{Características}
		Caja cúbica de $500 Mpc$, $512^3$ partículas.\\~\\
		\begin{itemize}
			\item 32 Procesadores (KIAS)
			\item 5 snapshots
			\item $\sim 3.4Gb$ de almacenamiento por snapshot
			\item 5 días de simulación
			\item Datos de Plank
			\item Variación de $\Omega_0\pm5\%$
		\end{itemize}
	\end{block}		
	
\end{frame}
%================================================%
\subsection{Resultados}
%================================================%
\begin{frame}{Comparación de Masa}
	\begin{columns}
		\begin{column}{0.5\textwidth}
			\begin{figure}[!h]
			\begin{center}
				\includegraphics[width=\textwidth]{im/logm-deltam-mas}
				%\caption{$\sigma_8=0.9$} 
				\label{fig:m1}
			\end{center}
		\end{figure}
		\end{column}
		
		\begin{column}{0.5\textwidth}
			\begin{figure}[!h]
			\begin{center}
				\includegraphics[width=\textwidth]{im/logm-deltam-menos}
				%\caption{$\sigma_8=0.7$} 
				\label{fig:m2}
			\end{center}
		\end{figure}
		\end{column}
	\end{columns}
\end{frame}
%================================================%
\begin{frame}{Comparación de Posición}
	\begin{columns}
		\begin{column}{0.5\textwidth}
			\begin{figure}[!h]
			\begin{center}
				\includegraphics[width=\textwidth]{im/logm-deltap-mas}
				%\caption{$\sigma_8=0.9$} 
				\label{fig:p1}
			\end{center}
		\end{figure}
		\end{column}
		
		\begin{column}{0.5\textwidth}
			\begin{figure}[!h]
			\begin{center}
				\includegraphics[width=\textwidth]{im/logm-deltap-menos}
				%\caption{$\sigma_8=0.7$} 
				\label{fig:p2}
			\end{center}
		\end{figure}
		\end{column}
	\end{columns}
\end{frame}
%================================================%
\begin{frame}{Comparación de Velocidad}
	\begin{columns}
		\begin{column}{0.5\textwidth}
			\begin{figure}[!h]
			\begin{center}
				\includegraphics[width=\textwidth]{im/logm-deltav-mas}
				%\caption{$\sigma_8=0.9$} 
				\label{fig:v1}
			\end{center}
		\end{figure}
		\end{column}
		
		\begin{column}{0.5\textwidth}
			\begin{figure}[!h]
			\begin{center}
				\includegraphics[width=\textwidth]{im/logm-deltav-menos}
				%\caption{$\sigma_8=0.7$} 
				\label{fig:v2}
			\end{center}
		\end{figure}
		\end{column}
	\end{columns}
\end{frame}
%================================================%
\begin{frame}{Distribuciones de cambios de velocidad y masa}
	\begin{columns}
		\begin{column}{0.5\textwidth}
			\begin{figure}[!h]
			\begin{center}
				\includegraphics[width=\textwidth]{im/deltammas}
				%\caption{$\sigma_8=0.9$} 
				\label{fig:mm1}
			\end{center}
		\end{figure}
		\end{column}
		
		\begin{column}{0.5\textwidth}
			\begin{figure}[!h]
			\begin{center}
				\includegraphics[width=\textwidth]{im/deltavmas}
				%\caption{$\sigma_8=0.7$} 
				\label{fig:vv1}
			\end{center}
		\end{figure}
		\end{column}
	\end{columns}
\end{frame}
%================================================%
\section{Conclusiones}
%================================================%
\begin{frame}{Conclusiones}
	\begin{block}{}
		\begin{itemize}
			\item Medición de parámetros ($\Omega$) por estimación de masas es inviable
			\item De los cambios observados en simulaciones sólo cambios en velocidad
			\item Cambios en velocidad sólo con experimentos de próxima generación (DESI)
			\item Deben ser capaces de discernir cambios en $\Delta z \sim 3.6\times 10^{-3}$
		\end{itemize}
	\end{block}
\end{frame}
%================================================%
	\begin{frame}
		\begin{center}
			\textcolor{ggray}{\Huge{MUCHAS GRACIAS}}
		\end{center}
		
	\end{frame}
%================================================%
	\begin{frame}[allowframebreaks]{Referencias}
		\begin{thebibliography}{}
			\bibitem{gadget} V. Springel. The cosmological simulation code gadget-2. Monthly Notices of the Royal Astronomical Society, 364, 2005.
			\bibitem{dm} Conservapedia. Dark Matter. [En línea] Disponible en: \url{http://www.conservapedia.com/images/thumb/4/44/DarkMatterNASA1.jpg/350px-DarkMatterNASA1.jpg}
			\bibitem{lss} Smithsonian Astrophysical Observatory. The Cosmic Infrared Background. [En línea] Disponible en:\url{http://www.cfa.harvard.edu/sites/www.cfa.harvard.edu/files/images/news//su201231.jpg}	
			\bibitem{hed} HETDEX - Hobby-Eberly Telescope Dark Energy Experiment \url{http://hetdex.org/}
			\bibitem{loeb} A. Loeb. How did the first stars and galaxies form? Princeton University Press, Princeton, NJ,
2010. 3, 4, 7

			
	\end{thebibliography}
%	\bibliographystyle{abbrv}
%	\bibliography{references}
	\end{frame}
%================================================%
\end{document}
